% Propriété de Luis L. Marques et Enzo Carré
\documentclass[12pt, a4paper]{report}

\usepackage[utf8]{inputenc}
\usepackage[T1]{fontenc}
\usepackage[english]{babel}
\usepackage[top = 2 cm, bottom = 2 cm, left = 2 cm, right = 2 cm]{geometry} % Page margins

\usepackage{amsmath}
\usepackage{graphicx}
\usepackage{algorithm,algorithmic}

\title{Object oriented programming}
\author{Luis L. marques\\Enzo Carré}
\date{2019/2020}

\begin{document}

\maketitle
\tableofcontents

\newpage
\section*{The game}
\addcontentsline{toc}{section}{The game}

\subsection*{Used tools}
\addcontentsline{toc}{subsection}{Used tools}

Because the course that ordered this project is based on the Java language, we had to use it along with JavaFX for the graphics.

As for the IDEs, we had a mix of Eclipse and IntelliJ users.

\subsection*{The basic version}
\addcontentsline{toc}{subsection}{The basic version}

The client ordered a basic and an advanced version of the game.

For the basic one, we had the following rules:

\begin{itemize}
    \item Only one type of troop. We chose to keep the knight.
    \item Only one level for castles.
    \item The troops don't evade castles or obstacles when moving.
    \item The troops don't leave the castle by the door.
    \item Per turn, castles produce troops instead of money.
    \item The bots don't do anything.
\end{itemize}

This was in order to have a first functional prototype without all the features, but that could be played by someone.

\subsection*{The advanced version}
\addcontentsline{toc}{subsection}{The advanced version}

For the advanced version, we took out the rules we specified in the previous subsection, and we added the following ones:

\begin{itemize}
    \item Many types of troops. We chose 4: Knight, Onager, Pikeman and Camel.
    \item Many levels for castles. We chose to have an infinity of levels by using mathematical formulas.
    \item The troops evade castles and obstacles when moving.
    \item The troops leave the castle by the door.
    \item Per turn, castles produce money, and money can be used to create troops and level up the castle.
    \item The bots do actions.
\end{itemize}

\noindent{
As bonus features, not included in contract, we added:
}

\begin{itemize}
    \item Walls around the castles.
    \item A troop type that can move money from one castle to another.
\end{itemize}

\subsection*{Known bugs}
\addcontentsline{toc}{subsection}{Known bugs}

Basic version:

\begin{itemize}
    \item Random crashes when launching the application.
    \item Randomly, the application won't start up.
    \item Sometimes, troops follow the path but are a bit off them when moving to another castle.
    \item After conquering a castle, sending troops again will make them appear twice, but the game only considers them once as we wanted.
    \item White bar on the right side of the window.
\end{itemize}

\noindent{
    Advanced version:
}

\begin{itemize}
    \item Random crashes when launching the application.
    \item White bar on the right side of the window.
\end{itemize}

\section*{How we did the game}
\addcontentsline{toc}{section}{How we did the game}

We were two on the project. In the first part, we just started coding on the same parts to find what we could do using JavaFX.

\subsection*{The common work}
\addcontentsline{toc}{subsection}{The common work}

The common work was the design of the interfaces and classes we would use as well as the beginning of the code, as it was shared between both versions.

\subsection*{The personal work}
\addcontentsline{toc}{subsection}{The personal work}

By the time we had separated our project into the basic and the advanced version, we also deletaged work.

I took on the basic version along with the javadoc and the documentation for both versions while Enzo took on the advanced version.

\section*{Personal Experience}
\addcontentsline{toc}{section}{Personal Experience}

\subsection*{Luis L. Marques}
\addcontentsline{toc}{subsection}{Luis L. Marques}

Personnally, I found this project a bit boring at the beginning, as it is not my type of game, and I wasn't familiar with Java.

Afterwards, I familiarised myself with JavaFX. Althought it doesn't resemble any other library I already used, it was quite easy to use.

Contract side, I felt the contract had some unprecisions about the players and the attack system. I also felt we lacked a bit of time, one or two more weeks would have been perfect for our team.

Code side, it was a great opportunity to discover and master a new language, and as for every other project, it was a great opportunity to develop team-play.

\subsection*{Enzo Carré}
\addcontentsline{toc}{subsection}{Enzo Carré}



\end{document}